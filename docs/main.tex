\documentclass{article}
\usepackage{graphicx}
\usepackage{amsmath}
\usepackage{amsfonts}
\usepackage{pl-syntax/pl-syntax}
\usepackage{mathpartir}
\usepackage{stmaryrd}

\title{Pirouette}
\author{Shruti Gupta \and Prashant Godhwani}
\date{September 2023}

\begin{document}

\maketitle

\section{Syntax}  \begin{syntax}

    \categoryFromSet[Locations]{\ell}{\mathcal{L}} 

    \category[Synchronization Labels]{d}
    \alternative{All Variables}

    \category[Choreography]{C}
 
    \alternative{\mathbf{()}}
    \alternative{X}
    \alternative{\ell . e}
    \alternative{\ell_1 . e \leadsto \ell_2 . x ; \ C }    
    \alternative{\textbf{if } \ell . e \textbf{ then } C_1 \textbf{ else } C_2}
    \\
    \alternative{\ell_1 [d] \leadsto \ell_2  ; \ C }  
    \alternative{ \textbf{let } \ell . x \coloneqq C_1 \textbf{ in } C_2}
    \alternative{\textbf{fun } F(X) \coloneqq C}
    \alternative{C_1 \  C_2}
    \\
    \alternative{(C_1, C_2)}
    \alternative{\textbf{fst } C}
    \alternative{\textbf{snd } C}
    \alternative{\textbf{inl } C}
    \alternative{\textbf{inr } C}
    \\
    \alternative{\textbf{match } C \textbf{ with } inl \  X \to C_1 \textbf{; } inr \  Y \to C_2}


    \category[Control Expressions]{E}
    \alternative{\ldots}
    \alternative{\mathbf{()}}
    \alternative{(E_1, E_2)}
    \alternative{\textbf{fst } E}
    \alternative{\textbf{snd } E}
    \alternative{\textbf{inl } E}
    \alternative{\textbf{inr } E}
    \\
    \alternative{\textbf{match } E \textbf{ with } inl \  X \to E_1 \textbf{; } inr \  Y \to E_2}


    \category[Choreographic Types]{\tau}
    \alternative{\textbf{unit}}
    \alternative{\ell.t}
    \alternative{\tau_1 \rightarrow \tau_2}
    \alternative{\tau_1 \times \tau_2}
    \alternative{\tau_1 + \tau_2}

    \category[Local Types]{t}
    \alternative{\textbf{unit}}
    \alternative{\textbf{int}}
    \alternative{\textbf{string}}
    \alternative{\textbf{bool}}
    \alternative{t_1 \rightarrow t_2}
    \alternative{t_1 \times t_2}
    \alternative{t_1 + t_2}
  \end{syntax}

\section{Type System}

\begin{mathpar}
\inferrule [ unit ]
{ \; }
{ \Gamma , \Delta \vdash \textsf{( ) : unit} } \and
\inferrule [ pair ]
{ \Gamma , \Delta \vdash \textsf{$C_1 : \tau_1$} \quad \Gamma , \Delta \vdash \textsf{$C_2 : \tau_2$} }
{ \Gamma , \Delta \vdash \textsf{$(C_1, C_2) : \tau_1 \times \tau_2$}  } \and
\inferrule [ fst ]
{ \Gamma , \Delta \vdash \textsf{$(C_1, C_2) : \tau_1 \times \tau_2$} }
{ \Gamma , \Delta \vdash \textsf{fst $(C_1, C_2) : \tau_1 $} } \and
\inferrule [ snd ]
{ \Gamma , \Delta \vdash \textsf{$(C_1, C_2) : \tau_1 \times \tau_2$} }
{ \Gamma , \Delta \vdash \textsf{snd $(C_1, C_2) : \tau_2 $} } \and
\inferrule [ inl ]
{ \Gamma , \Delta \vdash \textsf{$ C : \tau_1$} }
{ \Gamma , \Delta \vdash \textsf{inl $C : \tau_1 + \tau_2 $} } \and
\inferrule [ inr ]
{ \Gamma , \Delta \vdash \textsf{$ C : \tau_2$} }
{ \Gamma , \Delta \vdash \textsf{inr $C : \tau_1 + \tau_2 $} } \and

\inferrule [ match ]
{ \Gamma , \Delta \vdash \textsf{$ C : \tau_1+\tau_2$} \quad \Gamma , \Delta \textsf{, X : $\tau_1$} \vdash \textsf{$ C_1 : \tau_3$} \quad \Gamma , \Delta \textsf{, Y : $\tau_2$} \vdash \textsf{$ C_2 : \tau_3$}  }
{ \Gamma , \Delta \vdash \textsf{(match C with inl X $\to C_1$ ; inr Y $\to C_2$) : $ \tau_3 $} } 

\end{mathpar}

\section{Operational Semantics}
\subsection{Control Language}
\begin{mathpar}
\textsf{fst} (E_1, E_2)  \to E_1 \and
 \textsf{snd} (E_1, E_2)  \to E_2 \and
\textsf{(match \ inl \ E \ with \ inl X} \to E_1 \textsf{; inr Y} \to E_2) \to E_1\ [X \mapsto E] \and
\textsf{(match \ inr \ E \ with \ inl X} \to E_1 \textsf{; inr Y} \to E_2) \to E_2\ [Y \mapsto E]
\end{mathpar}
\\
\subsection{Choreography}
\begin{mathpar}
\textsf{fst} (C_1, C_2)  \to C_1 \and
 \textsf{snd} (C_1, C_2)  \to C_2 \and
\textsf{(match \ inl \ C \ with \ inl X} \to C_1 \textsf{; inr Y} \to C_2) \to C_1\ [X \mapsto C] \and
\textsf{(match \ inr \ C \ with \ inl X} \to C_1 \textsf{; inr Y} \to C_2) \to C_2\ [Y \mapsto C]
\end{mathpar}

\section{Glossary}
$$
\ell \textsf{ involved in } \tau = 
    \begin{array}{l}
    \ell \in locs (\tau)
    \end{array}
$$
$$
 \ell \in locs (\tau) = \textsf{ getLoc is a function that recursively traverses over } \tau \textsf{ to construct } locs (\tau)
$$

$$
 locs (\tau) = \left\{
    \begin{array}{ll}
    \phi & \textsf{if } \tau = \textbf{unit}\\
    \{\ell\} & \textsf{if } \tau = \ell.e\\
        \textsf{getLoc } \tau_1 \cup \textsf{getLoc } \tau_2 & \textsf{if } \tau = \tau_1 \to \tau_2 \textsf{ or } \tau_1 + \tau_2 \textsf{ or } \tau_1 \times \tau_2 \\
    \end{array}
\right.
$$

\section{Endpoint Projection}

$$
 \llbracket     (C_1, \ C_2) \rrbracket _\ell = \left\{
    \begin{array}{l l}
    (\llbracket     C_1 \rrbracket _\ell ,\  \llbracket     C_2 \rrbracket _\ell)  & 
    \textsf{if } (C_1, C_2) : \tau_1 \times \tau_2 \textsf{ and } \ell \textsf{ is involved in } C_1,  C_2, \tau_1, \ \tau_2 \\
    \begin{array}{l}
    \textsf{let x = } \llbracket C_1 \rrbracket _\ell \textsf{ in} \\
    \textsf{let \_ = } \llbracket C_2 \rrbracket _\ell \textsf{ in x}
    \end{array} & \textsf{if } (C_1, C_2) : \tau_1 \times \tau_2 \textsf{ and } \ell \textsf{ is involved in } C_1, C_2 \textsf{ and } \tau_1 \textsf{ but not in } \tau_2  \\
    \llbracket     C_1 \rrbracket _\ell; \llbracket     C_2 \rrbracket _\ell & \textsf{if } (C_1, C_2) : \tau_1 \times \tau_2 \textsf{ and } \ell \textsf{ is involved in }  C_1 \textsf{ or }  C_2 \\ 
    \textbf{()} & \textsf{otherwise}
    \end{array}
\right.
$$

$$
 \llbracket  \textbf{fst } C \rrbracket _\ell = \left\{
    \begin{array}{ll}
    \textbf{fst }\llbracket     C \rrbracket _\ell & \textsf{if } C : \tau_1 \times \tau_2 \textsf{ and } \ell \textsf{ is involved in } C,  \tau_1 \textsf{ and } \tau_2\\
    \llbracket C \rrbracket _\ell & \textsf{if } C : \tau_1 \times \tau_2 \textsf{ and } \ell \textsf{ is involved in } C  \\
    \textbf{()} & \textsf{otherwise}
    \end{array}
\right.
$$

$$
 \llbracket  \textbf{snd} C \rrbracket _\ell = \left\{
    \begin{array}{ll}
    \textbf{snd }\llbracket     C \rrbracket _\ell & \textsf{if } C : \tau_1 \times \tau_2 \textsf{ and } \ell \textsf{ is involved in } C,  \tau_1 \textsf{ and } \tau_2\\
    \llbracket C \rrbracket _\ell & \textsf{if } C : \tau_1 \times \tau_2 \textsf{ and } \ell \textsf{ is involved in } C  \\
    \textbf{()} & \textsf{otherwise}
    \end{array}
\right.
$$

$$
 \llbracket  \textbf{inl } C \rrbracket _\ell = \left\{
    \begin{array}{ll}
    \textbf{inl }\llbracket     C \rrbracket _\ell & \textsf{if } C : \tau_1 \textsf{ and } \ell \textsf{ is involved in } C \textsf{ and } \tau_1 \\
    \llbracket     C \rrbracket _\ell & \textsf{if } C : \tau_1 \textsf{ and } \ell \textsf{ is involved in } C \textsf{ but not in } \tau_1 \\
    \textbf{()} & \textsf{if } C : \tau_1 \textsf{ and } \ell \textsf{ is not involved in } C \textsf{ and } \tau_1
    \end{array}
\right.
$$

$$
 \llbracket  \textbf{inr } C \rrbracket _\ell = \left\{
    \begin{array}{ll}
    \textbf{inr }\llbracket     C \rrbracket _\ell & \textsf{if } C : \tau_2 \textsf{ and } \ell \textsf{ is involved in } C \textsf{ and } \tau_2 \\
    \llbracket     C \rrbracket _\ell & \textsf{if } C : \tau_2 \textsf{ and } \ell \textsf{ is involved in } C \textsf{ but not in } \tau_2 \\
    \textbf{()} & \textsf{if } C : \tau_2 \textsf{ and } \ell \textsf{ is not involved in } C \textsf{ and } \tau_2
    \end{array}
\right.
$$

$$\begin{array}{l}
 \llbracket  \textbf{match } C \textbf{ in inl X } \to C_1 \textbf{; inr Y } \to C_2 \rrbracket _\ell\\ \\ \;\;= \left\{
    \begin{array}{l l}
    \textbf{match }\llbracket C \rrbracket _\ell \textbf{ in inl X } \to \llbracket C_1 \rrbracket _\ell \textbf{; inr Y } \to \llbracket C_2 \rrbracket _\ell & 
    \begin{array}{l}
    \textsf{if } C : \tau_1 + \tau_2 \textsf{ and } \ell \textsf{ is involved in} \\ \textsf{both } \tau_1 \textsf{ and }\tau_2 
    \end{array}    \\
    \llbracket C \rrbracket _\ell \textbf{; } \llbracket C_1 \rrbracket _\ell \sqcup \llbracket C_2 \rrbracket _\ell & 
    \begin{array}{l}
    \textsf{if } C : \tau_1 + \tau_2 \textsf{ and } \ell \textsf{ is involved in } \\ \tau_1 \textsf{ or } \tau_2 \textsf{ or } C
    \end{array}    \\
    \llbracket C_1 \rrbracket _\ell \sqcup \llbracket C_2 \rrbracket _\ell & 
    \begin{array}{l}
    \textsf{if } C : \tau_1 + \tau_2 \textsf{ and } \ell \textsf{ is not involved in} \\ C, \ \tau_1 \textsf{ and } \tau_2 
    \end{array}
    \end{array}
\right.
\end{array}
$$

\section{Type Projection} 
$$
 \llbracket  \textbf{unit} \rrbracket _\ell =
    \begin{array}{ll}
    \textbf{unit}
    \end{array}
$$

$$
 \llbracket  \ell_1 . t \rrbracket _{\ell_2} = \left\{
    \begin{array}{ll}
    t & \textsf{if } \ell_1 = \ell_2 \\
    \textbf{unit} & \textsf{otherwise}
    \end{array}
\right.
$$

$$
 \llbracket  \tau_1 \to \tau_2 \rrbracket _\ell = \left\{
    \begin{array}{ll}
    \llbracket  \tau_1 \rrbracket _\ell \to \llbracket  \tau_2 \rrbracket _\ell & \textsf{if } \ell \textsf{ is involved in } \tau_1 \textsf{ or } \tau_2 \textsf{ or both}\\
    \textbf{unit} & \textsf{otherwise}
    \end{array}
\right.
$$

$$
 \llbracket  \tau_1 + \tau_2 \rrbracket _\ell = \left\{
    \begin{array}{ll}
    \llbracket  \tau_1 \rrbracket_\ell + \llbracket  \tau_2 \rrbracket _\ell & \textsf{if } \ell \textsf{ is involved in } \tau_1 \textsf{ or } \tau_2 \textsf{ or both}\\
    \textbf{unit} & \textsf{otherwise}
    \end{array}
\right.
$$

$$
 \llbracket  \tau_1 \times \tau_2 \rrbracket _\ell = \left\{
    \begin{array}{ll}
    \llbracket  \tau_1 \rrbracket_\ell \times \llbracket  \tau_2 \rrbracket _\ell & \textsf{if } \ell \textsf{ is involved in } \tau_1 \textsf{ and } \tau_2 \\
    \llbracket  \tau_1 \rrbracket_\ell & \textsf{if } \ell \textsf{ is involved in } \tau_1 \textsf{ but not } \tau_2 \\
    \llbracket  \tau_2 \rrbracket_\ell & \textsf{if } \ell \textsf{ is involved in } \tau_2 \textsf{ but not } \tau_1 \\
    \textbf{unit} & \textsf{otherwise}
    \end{array}
\right.
$$

\section{Lemmas}
$$
    \begin{array}{ll}
    \textbf{Lemma 1 : If } \ell \textbf{ is not involved in } \tau \textbf{ then } \llbracket  \tau \rrbracket_\ell = \textbf{unit}
    \\
    \textbf{Proof : } \textsf{By Induction on } \tau \\
    \\
     \textbf{Case } \tau = \textbf{unit : } \\
   \quad \textsf{From type projection we know, }   \llbracket 
   \textbf{unit} \rrbracket _\ell = \textsf{unit} \\
   \\
   \textbf{Case } \tau = \ell.t\textbf{ : } \\
    \quad \textsf{Here we know that, } \ell_1 \neq \ell \, \  \textsf{as } \ell \textsf{ is not involved in } \tau \\
    \quad \textsf{so using type projection for } \llbracket 
   \ell_1.t\rrbracket _\ell \\ \quad \textsf{we can say, }
   \llbracket 
   \ell_1.t\rrbracket _\ell = \textbf{unit } \\
   \\
   \textbf{Case } \tau = \llbracket  \tau_1 \to \tau_2 \rrbracket _\ell \textbf{: } \\
   \quad \textsf{By IH, } \ell \textsf{ is not involved in } \tau_1 \textsf{ and } \ell \textsf{ is not involved in } \tau_2\\
    \quad \textsf{so using type projection for } \llbracket 
   \tau_1 \to \tau_2\rrbracket _\ell \\ \quad \textsf{we can say, }
   \llbracket 
   \tau_1 \to \tau_2\rrbracket _\ell = \textbf{unit } \\
   \\
   \textbf{Case } \tau = \llbracket  \tau_1 + \tau_2 \rrbracket _\ell \textbf{: } \\
   \quad \textsf{By IH, } \ell \textsf{ is not involved in } \tau_1 \textsf{ and } \ell \textsf{ is not involved in } \tau_2\\
    \quad \textsf{so using type projection for } \llbracket 
   \tau_1 + \tau_2\rrbracket _\ell \\ \quad \textsf{we can say, }
   \llbracket 
   \tau_1 + \tau_2\rrbracket _\ell = \textbf{unit } \\
   \\
   \textbf{Case } \tau = \llbracket  \tau_1 \times \tau_2 \rrbracket _\ell \textbf{: } \\
   \quad \textsf{By IH, } \ell \textsf{ is not involved in } \tau_1 \textsf{ and } \ell \textsf{ is not involved in } \tau_2\\
    \quad \textsf{so using type projection for } \llbracket 
   \tau_1 \times \tau_2\rrbracket _\ell \\ \quad \textsf{we can say, }
   \llbracket 
   \tau_1 * \tau_2\rrbracket _\ell = \textbf{unit } \\
   
    
    \\
    \\
    \textbf{Lemma 2 : If } \ell \notin locs(C) \textbf{ then }\llbracket  C \rrbracket_\ell = \textbf{()}
    \\
    \\
    \textbf{Lemma 3 : If } \vdash C : \tau, \textbf{ then } \vdash \llbracket  C \rrbracket_\ell : \llbracket  \tau \rrbracket_\ell
    \end{array}
$$


\section{ALPS Syntax}  \begin{syntax}

    \categoryFromSet[Locations]{\ell}{\mathcal{L}} 

    \category[Synchronization Labels]{d}
      \alternative{\textbf{All Variables}}

    \category[Integers]{i}
    \alternative{\textbf{All Integers}}

    \category[Strings]{s}
    \alternative{\textbf{All Strings}}

     \category[Boolean]{b}
    \alternative{true}
    \alternative{false}

        \category[Variables]{x}
    \alternative{\textbf{All Variables}}

    \category[Binary Operations]{binop}
    \alternative{+}
    \alternative{-}
    \alternative{*}
    \alternative{/}
    \alternative{=}
    \alternative{<=}
    \alternative{>=}
    \alternative{!=}
    \alternative{>}
    \alternative{<}
    \alternative{\&\&}
    \alternative{\parallel}

     \category[Value]{val}
    \alternative{i}
    \alternative{b}
    \alternative{s}
    
    \category[Local Types]{t}
    \alternative{\textbf{unit}}
    \alternative{\textbf{int}}
    \alternative{\textbf{string}}
    \alternative{\textbf{bool}}
    
     \category[Local Expressions]{e}
     \alternative{()}
    \alternative{val}
    \alternative{x}
    \alternative{e_1 \ binop \ e_2}
    \alternative{\textbf{let} \ x = e_1 \ in \ e_2}
    \alternative{(e_1, e_2)} 
    \alternative{\textbf{fst } e} \\
    \alternative{\textbf{snd } e}
    \alternative{\textbf{left } e}
    \alternative{\textbf{right } e}
    \alternative{ \textbf{ match } e \textbf{ with } \ [ \ | \ p \to e_1\ ]^*}


    \category[Comments]{comments}
    \alternative{\textbf{- -}}
    \alternative{\textbf{\{-} \ \ \textbf{-\}}}

    \category[Declarations]{D}
    \alternative{F : \tau_1 \to \tau_2}
    \alternative{X : \tau}
    \alternative{\ell.x : \ell.t}
    \alternative{\textbf{type} \ name = \tau}

    \category[Assignment]{A}
    \alternative{X = C}
    \alternative{F \ P_1 \dots P_n = C}
     \alternative{\ell .x = C}
% decl_block = . (empty) | D decl_block (declaration) | D; A decl_block (assignment)
    \category[Declaration Block]{decl\_block}
    \alternative{\cdot}
    \alternative{D \ decl\_block }
    \alternative{A \ decl\_block}
   
    \category[Local Patterns]{p}
    \alternative{\_}
    \alternative{val}
    \alternative{x}
    \alternative{(p_1, p_2)}
    \alternative{left \ p}
    \alternative{right \ p}
    
    \category[Patterns]{P}
    \alternative{\_}
    \alternative{x}
    \alternative{(P_1, P_2)}
    \alternative{\ell.p}
    \alternative{left \ P}
    \alternative{right \ P}


       \category[Choreographic Types]{\tau}
    \alternative{\textbf{unit}}
    \alternative{\ell.t}
    \alternative{\tau_1 \rightarrow \tau_2}
    \alternative{\tau_1 \times \tau_2}
    \alternative{\tau_1 + \tau_2}
    
    \category[Choreography]{C}
 
    \alternative{\mathbf{()}}
    \alternative{X}
    \alternative{\ell . e}
    \alternative{\ell_1 . e \leadsto \ell_2 . x ; \ C }    
    \alternative{C \leadsto \ell }  
    \\
    \alternative{\textbf{if } C_1 \textbf{ then } C_2 \textbf{ else } C_3}
    \\
    \alternative{\ell_1 [d] \leadsto \ell_2  ; \ C }  
    % \alternative{ \textbf{let } \ell . x = C_1 \textbf{ in } C_2}
    \alternative{ \textbf{let } decl\_block \textbf{ in } C}
    \\
    \alternative{\textbf{fun } X \to C}
    \alternative{C_1 \  C_2}
    \\
    \alternative{(C_1, C_2)}
    \alternative{\textbf{fst } C}
    \alternative{\textbf{snd } C}
    \alternative{\textbf{left } C}
    \alternative{\textbf{right } C}
    \\
    \alternative{ \textbf{ match } C \textbf{ with } \ [ \ | \ P \to C_1\ ]^*}

        \category[Network Types]{t_N}
    \alternative{t}
    \alternative{t_N_1 \rightarrow t_N_2}
    \alternative{t_N_1 \times t_N_2}
    \alternative{t_N_1 + t_N_2}
    
    \category[Network Expressions]{E}
    \alternative{X}
    \alternative{\mathbf{()}}
    \alternative{\textbf{fun } X \to E}
    \alternative{E_1 \ E_2 }
    \alternative{\textsf{ret}(e)}
    \\
    \alternative{\textsf{let ret}(x) = E_1 \ \textsf{in} \ E_2}
    \alternative{\textsf{send } e \textsf{ to } \ell ; \  E}
    \alternative{\textsf{receive } x \textsf{ from } \ell ; \  E}
    \\
    \alternative{\textsf{if } E_1 \textsf{ then } E_2 \textsf{ else }  E_3}
    \alternative{\textsf{choose } d \textsf{ for } \ell ; \  E}
    \\
    % \alternative{\textsf{allow } \ell \textsf{ choice } | \ L \Rightarrow E_1 \ | \ R \Rightarrow E_2}
     \alternative{\textsf{allow } \ell \textsf{ choice } [ \ | \ d \to E\ ]^*}
    \alternative{(E_1, E_2)}
    \\
    \alternative{\textbf{fst } E}
    \alternative{\textbf{snd } E}
    \alternative{\textbf{left } E}
    \alternative{\textbf{right } E}
    \\
    \alternative{ \textbf{ match } E \textbf{ with } \ [ \ | \ p \to E_1\ ]^*}

    \category[Program]{\rho}
    \alternative{decl\_block \qed{} }

    
  \end{syntax}


\end{document}
