\documentclass{article}
% \documentclass{notes}
\usepackage{graphicx} % Required for inserting images
\usepackage{amsmath}
\usepackage{amsfonts}
\usepackage{pl-syntax/pl-syntax}
\usepackage{mathpartir}
\usepackage{stmaryrd}
 


\title{Pirouette - Theory}
\author{\textbf{Shruti Gupta} \and \textbf{Prashant Godhwani}}
\date{February 2024}


\begin{document}

\maketitle

\section{Syntax}  \begin{syntax}

    \categoryFromSet[Locations]{\ell}{\mathcal{L}} 

    \category[Synchronization Labels]{d}
    \alternative{L}
    \alternative{R}

     \category[Binary Operations]{\odot}
    \alternative{+}
    \alternative{-}
    \alternative{*}
    \alternative{/}
    \alternative{=}
    \alternative{<=}
    \alternative{>=}
    \alternative{!=}
    \alternative{>}
    \alternative{<}
    \alternative{\&\&}
    \alternative{\parallel}

    \category[Choreography]{C}
    % Each alternative is introduced on its own line.
    \alternative{\mathbf{()}}
    \alternative{X}
    \alternative{\ell . e}
    \alternative{C \leadsto \ell}    
    \alternative{\textbf{if } C \textbf{ then } C_1 \textbf{ else } C_2}
    \\
    \alternative{\ell_1 [d] \leadsto \ell_2  ; \ C }  
    \alternative{ \textbf{let } X := C_1 \textbf{ in } C_2}
    \alternative{\textbf{fun }F \ X \Rightarrow C}
    \alternative{C_1 \  C_2}
    \\
    \alternative{(C_1, C_2)}
    \alternative{\textbf{fst } C}
    \alternative{\textbf{snd } C}
    \alternative{\textbf{left } C}
    \alternative{\textbf{right } C}
    \\
    \alternative{\textbf{match } C \textbf{ with  left} \  X \Rightarrow C_1 \textbf{;  right} \  Y \Rightarrow C_2}



    % You can add vertical spacing between categories to visually group them.
    % \separate
    % You can pass the amount of space explicitly if you want to manually control it:
    % \separate[5ex]


     \category[Local Expressions]{e}
     \alternative{()}
    \alternative{num}
    \alternative{x}
    \alternative{e_1 \ binop \ e_2}
    \alternative{\textbf{let} \ x := e_1 \ in \ e_2}
    \alternative{(e_1, e_2)} 
    \alternative{\textbf{fst } e} \\
    \alternative{\textbf{snd } e}
    \alternative{\textbf{left } e}
    \alternative{\textbf{right } e}
    \alternative{\textbf{match } e \textbf{ with  left} \  x \Rightarrow e_1 \textbf{;  right} \  y \Rightarrow e_2}
    
    \category[Network Expressions]{E}
    \alternative{X}
    \alternative{\mathbf{()}}
    \alternative{\textbf{fun}\ F\ X \Rightarrow E}
    \alternative{E_1 \ E_2 }
    \alternative{\textsf{ret}(e)}
    \\
    \alternative{\textsf{let ret}(x) := E_1 \ \textsf{in} \ E_2}
    \alternative{\textsf{send } e \textsf{ to } \ell ; \  E}
    \alternative{\textsf{receive } x \textsf{ from } \ell ; \  E}
    \\
    \alternative{\textsf{if } E_1 \textsf{ then } E_2 \textsf{ else }  E_3}
    \alternative{\textsf{choose } d \textsf{ for } \ell ; \  E}
    \\
    \alternative{\textsf{allow } \ell \textsf{ choice L} \Rightarrow E_1 \ ; \ \textsf{R} \Rightarrow E_2}
    \alternative{(E_1, E_2)}
    \\
    \alternative{\textbf{fst } E}
    \alternative{\textbf{snd } E}
    \alternative{\textbf{left } E}
    \alternative{\textbf{right } E}
    \\
    \alternative{\textbf{match } E \textbf{ with  left} \  X \Rightarrow E_1 \textbf{;  right} \  Y \Rightarrow E_2}

    \category[Choreographic Types]{\tau}
    \alternative{\textbf{unit}}
    \alternative{\ell.t}
    \alternative{\tau_1 \rightarrow \tau_2}
    \alternative{\tau_1 \times \tau_2}
    \alternative{\tau_1 + \tau_2}

    \category[Local Types]{t}
    \alternative{\textbf{unit}}
    \alternative{\textbf{int}}
    \alternative{\textbf{bool}}
    \alternative{\textbf{string}}
    \alternative{t_1 \times t_2}
    \alternative{t_1 + t_2}


    \category[Network Types]{T}
    \alternative{\textbf{unit}}
    \alternative{\boxed{t}}
    \alternative{T_1 \rightarrow T_2}
    \alternative{T_1 \times T_2}
    \alternative{T_1 + T_2}

  \end{syntax}

\section{Type System}
\subsection{Local Language}
\begin{mathpar}
\inferrule [ Loc - unit ]
{ \; }
{ \Gamma \vdash \textsf{( ) : unit} } \and
\inferrule [ Loc - var ]
{ x : t \in \Gamma }
{ \Gamma \vdash x : t  } \and
\inferrule [ Loc - binop mat ] 
{ \Gamma \vdash e_1 : int \quad \Gamma \vdash e_2 : int \quad \odot \in \{+,-,\times, /\}} 
{ \Gamma \vdash e_1 \odot e_2 : int} \and
\inferrule [ Loc - binop comp ] 
{ \Gamma \vdash e_1 : int \quad \Gamma \vdash e_2 : int \quad \odot \in \{=, !=, <, >, <=, >=\}} 
{ \Gamma \vdash e_1 \odot e_2 : bool} \and
\inferrule [ Loc - binop bool ] 
{ \Gamma \vdash e_1 : bool \quad \Gamma \vdash e_2 : bool \quad \odot \in \{\& \&, \parallel\}} 
{ \Gamma \vdash e_1 \odot e_2 : bool} \and
\inferrule [ Loc - def ]
{ \Gamma \vdash e_1 : t \quad \Gamma, x : t \vdash e_2 : t_2 }
{ \Gamma ;\Delta \vdash \textsf{let}\ x := e_1 \ \textsf{in}\ e_2 : t_2  } \and
\inferrule [ Loc - pair ]
{ \Gamma \vdash e_1 : t_1 \quad \Gamma \vdash e_2 : t_2}
{ \Gamma \vdash (e_1, e_2) : t_1 \times t_2 } \\
\inferrule [ Loc - fst ]
{ \Gamma \vdash e : t_1 \times t_2 }
{ \Gamma \vdash \textsf{fst} \  e : t_1 } \and
\inferrule [ Loc - snd ]
{ \Gamma \vdash e : t_1 \times t_2 }
{ \Gamma \vdash \textsf{snd} \ e : t_2 } \\
\inferrule [ Loc - left ]
{ \Gamma \vdash e_1 : t_1 }
{ \Gamma \vdash \textsf{left} \ e_1 : t_1 + t_2 } \and
\inferrule [ Loc - right ]
{ \Gamma \vdash e_2 : t_2 }
{ \Gamma \vdash \textsf{right} \ e_2 : t_1 + t_2 } \and
\inferrule [ Loc - match ]
{ \Gamma \vdash e : t_1+t_2 \quad \Gamma , \ x : t_1 \vdash e_1 : t_3 \quad \Gamma , \ y : t_2 \vdash e_2 : t_3  }
{ \Gamma \vdash (\textsf{match e with left} \ x \Rightarrow e_1 ; \textsf{right} \ y \Rightarrow e_2) :t_3  } 

\end{mathpar}


\subsection{Network Language}
\begin{mathpar}
\inferrule [ Network - unit ]
{ \; }
{ \Gamma ;\Delta \vdash \textsf{( ) : unit} } \and
\inferrule [ Network - var ]
{ X : T \in \Delta }
{ \Gamma ;\Delta \vdash X : T } \and
\inferrule [ ret ]
{ \Gamma ;\Delta\vdash e : t }
{ \Gamma ;\Delta\vdash \textsf{ret} \ (e) : \boxed{t} } \and
\inferrule [ Network - fun ]
{ \Gamma ;\Delta , X : T_1, F: T_1 \to T_2 \vdash E : T_2 }
{ \Gamma ;\Delta \vdash \textsf{fun} \ F \ X \Rightarrow E : T_1 \to T_2  } \and
\inferrule [ Network - app ]
{ \Gamma ;\Delta \vdash E_1 : T_1 \to T_2 \quad \Gamma ;\Delta \vdash E_2 : T_1 }
{ \Gamma ;\Delta \vdash E_1 E_2 : T_2  } \and
\inferrule [ Network - if ]
{ \Gamma ;\Delta \vdash E_1 : \boxed{bool} \quad \Gamma ;\Delta \vdash E_2 : T_2 \quad \Gamma ;\Delta \vdash E_3 : T_2 }
{ \Gamma ;\Delta \vdash \textsf{if} \ E_1 \ \textsf{then}\ E_2 \ \textsf{else}\ E_3: T_2  } \and
\inferrule [ Network - def ]
{ \Gamma ;\Delta \vdash E_1 : \boxed{t} \quad \Gamma, x : t ;\Delta \vdash E_2 : T_2 }
{ \Gamma ;\Delta \vdash \textsf{let ret}\ (x) = E_1\ \textsf{in}\ E_2 : T_2  } \\
\inferrule [ Network - send ]
{ \Gamma ;\Delta \vdash e : t \quad \Gamma ;\Delta \vdash E : T }
{ \Gamma ;\Delta \vdash \textsf{send $e$ to}\ \ell; E : T  } \and
\inferrule [ Network - rcv ]
{ \Gamma , x : t ;\Delta \vdash E : T }
{ \Gamma ;\Delta \vdash \textsf{receive $x$ from}\ \ell; E : T  } \and
\inferrule [ Network - choose ]
{\Gamma ;\Delta \vdash E : T }
{ \Gamma ;\Delta \vdash \textsf{choose $d$ for}\ \ell; E : T  } \and
\inferrule [ Network - allow ]
{ \Gamma ;\Delta \vdash E_1 : T \quad \Gamma ;\Delta \vdash E_2 : T }
{ \Gamma ;\Delta \vdash (\textsf{allow $\ell $ choice L} \Rightarrow E_1 \; ; \textsf{R} \Rightarrow E_2) : T  } \and
\inferrule [ Network - pair ]
{ \Gamma ;\Delta \vdash E_1 : T_1 \quad \Gamma ;\Delta \vdash E_2 : T_2 }
{ \Gamma ;\Delta \vdash (E_1, E_2) : T_1 \times T_2 } \and
\inferrule [ Network - fst ]
{ \Gamma ;\Delta \vdash E : T_1 \times T_2 }
{ \Gamma ;\Delta \vdash \textsf{fst}\ E : T_1 } \and
\inferrule [ Network - snd ]
{ \Gamma ;\Delta \vdash E : T_1 \times T_2 }
{ \Gamma ;\Delta \vdash \textsf{snd}\ E : T_2 } \and
\inferrule [ Network - left ]
{ \Gamma ;\Delta \vdash E_1 : T_1 }
{ \Gamma ;\Delta \vdash \textsf{left}\ E_1 : T_1 + T_2 } \and
\inferrule [ Network - right ]
{ \Gamma ;\Delta \vdash  E_2 : T_2 }
{ \Gamma ;\Delta \vdash \textsf{right}\ E_2 : T_1 + T_2 } \and

\inferrule [ Network - match ]
{ \Gamma \vdash E : T_1+T_2 \quad \Gamma;\Delta, X : T_1 \vdash  E_1 : T_3 \quad \Gamma;\Delta , Y : T_2 \vdash E_2 : T_3  }
{ \Gamma \vdash (\textsf{match}\ E\ \textsf{with left}\; X \Rightarrow E_1 ; \textsf{right}\; Y \Rightarrow E_2) : T_3 } 

\end{mathpar}

\subsection{Choreography}

\begin{mathpar}
\inferrule [ unit ]
{ \; }
{ \Gamma ; \Delta \vdash \textsf{( ) : unit} } \and
\inferrule [ var ]
{ X : \tau \in \Delta }
{ \Gamma ;\Delta \vdash X : \tau } \and
\inferrule [ done ]
{ \Gamma_{|\ell} \vdash e : t }
{ \Gamma ; \Delta \vdash \ell.e : \ell.t } \and
\inferrule [ send ]
{ \Gamma ; \Delta \vdash C : \ell.t}
{ \Gamma;\Delta \vdash C \leadsto \ell_2 : \ell_2.t} \and
\inferrule [ sync ]
{ \Gamma;\Delta \vdash C : \tau}
{ \Gamma;\Delta \vdash \ell_1[d] \leadsto \ell_2 ; C : \tau} \and
\inferrule [ if ]
{ \Gamma;\Delta \vdash C_1 : \ell.bool \quad \Gamma;\Delta\vdash C_2 : \tau_2 \quad \Gamma;\Delta \vdash C_3 : \tau_2 }
{ \Gamma;\Delta \vdash \textsf{if } C_1 \textsf{ then } C_2 \textsf{ else } C_3: \tau_2  } \and
\inferrule [ def ]
{ \Gamma;\Delta \vdash C_1 : \ell.t \quad \Gamma;\Delta, X : \ell.t \vdash C_2 : \tau_2 }
{ \Gamma;\Delta \vdash \textsf{let } X = C_1 \textsf{ in } C_2 : \tau_2 } \and
\inferrule [ fun ]
{ \Gamma ;\Delta , X : \tau_1, F : \tau_1 \to \tau_2\vdash C : \tau_2 }
{ \Gamma ;\Delta \vdash \textsf{fun}\ F \ X \Rightarrow C : \tau_1 \to \tau_2  } \and
\inferrule [ app ]
{ \Gamma;\Delta \vdash C_1 : \tau_1 \to \tau_2 \quad \Gamma;\Delta \vdash C_2 : \tau_1 }
{ \Gamma;\Delta \vdash C_1 C_2 : \tau_2  } \and
\inferrule [ pair ]
{ \Gamma ; \Delta \vdash C_1 : \tau_1 \quad \Gamma ; \Delta \vdash C_2 : \tau_2 }
{ \Gamma ; \Delta \vdash (C_1, C_2) : \tau_1 \times \tau_2  } \and
\inferrule [ fst ]
{ \Gamma ; \Delta \vdash C : \tau_1 \times \tau_2 }
{ \Gamma ; \Delta \vdash \textsf{fst}\ C : \tau_1 } \and
\inferrule [ snd ]
{ \Gamma ; \Delta \vdash C : \tau_1 \times \tau_2 }
{ \Gamma ; \Delta \vdash \textsf{snd}\ C : \tau_2 } \\
\inferrule [ left ]
{ \Gamma ; \Delta \vdash C : \tau_1 }
{ \Gamma ; \Delta \vdash \textsf{left}\ C : \tau_1 + \tau_2 } \and
\inferrule [ right ]
{ \Gamma ; \Delta \vdash C : \tau_2 }
{ \Gamma ; \Delta \vdash \textsf{right}\ C : \tau_1 + \tau_2 } \and

\inferrule [ match ]
{ \Gamma ; \Delta \vdash C : \tau_1+\tau_2 \quad \Gamma ; \Delta , X : \tau_1 \vdash C_1 : \tau_3 \quad \Gamma ; \Delta , Y : \tau_2 \vdash C_2 : \tau_3  }
{ \Gamma ; \Delta \vdash (\textsf{match}\ C \ \textsf{with left}\ X \Rightarrow C_1 ; \textsf{right}\ Y \Rightarrow C_2) : \tau_3 } 

\end{mathpar}


\section{Operational Semantics}
\subsection{Local Language}
\begin{syntax}
     \category[Local values]{v}
     \alternative{()}
    \alternative{num}
    \alternative{v_1 \odot v_2}
    \alternative{(v_1, v_2)} 
    \alternative{\textbf{left } v}
    \alternative{\textbf{right } v}
\end{syntax}

\begin{mathpar}
\inferrule  [binop 1 ]
{ e_1 \to_l e'_1 }
{ e_1 \odot  e_2 \to_l  e'_1  \Tilde{\odot} e_2  } \and
\inferrule [ binop 2]
{ e_2 \to_l e'_2 }
{ v \odot e_2 \to_l v \Tilde{\odot} e'_2  } \and
\inferrule [binop 3 ]
{    }
{ v_1 \odot v_2 \to_l v_1 \Tilde{\odot} v_2 } \and

\inferrule[Loc - let 1]
{e_1 \to_l e'_1}
{\textsf{let $x := e_1$ in $e_2$} \to_l \textsf{let $x := e'_1$ in $e_2$} } \and

\inferrule[Loc - let 2]
{ }
{\textsf{let $ x := v$ in $e$} \to_l e[x \mapsto v] } \and

\inferrule [loc - pair 1]
{ e_1 \to_l e'_1 }
{ (e_1, e_2) \to_l (e'_1, e_2)  } \and
\inferrule [loc - pair 2 ]
{  e_2 \to_l e'_2    }
{ (v, e_2) \to_l (v, e'_2)  } \\
\inferrule [loc - fst ]
{ e \to_l e' }
{ \textsf{fst } e \to_l \textsf{fst } e' } \and
\inferrule [loc - pair elim 1 ]
{ }
{ \textsf{fst } (v_1, v_2) \to_l v_1 }\and
\inferrule [loc - snd ]
{ e \to_l e' }
{ \textsf{snd } e \to_l \textsf{snd } e' } \and
\inferrule [loc - pair elim 2 ]
{  }
{ \textsf{snd } (v_1, v_2) \to_l v_2 } \\
\inferrule [loc - left ]
{ e \to_l e' }
{ \textsf{left } e \to_l \textsf{left } e' } \and
\inferrule [ loc - right]
{ e \to_l e' }
{ \textsf{right } e \to_l \textsf{right } e' } \and
\inferrule [ Loc - match ]
{ e \to_l e' }
{ \textsf{(match e with left x $\Rightarrow e_1$ ; right y $\Rightarrow e_2$)} \to_l \textsf{(match e' with left x $\Rightarrow e_1$ ; right y $\Rightarrow e_2$)} } \and
\inferrule [loc - Sum elim 1 ]
{ }
{ \textsf{(match (left v) with left x $\Rightarrow e_1$ ; right y $\Rightarrow e_2$)} \to_l e_1\ [x \mapsto v] } \and
\inferrule [loc - sum elim 2 ]
{ }
{ \textsf{(match (right v) with left x $\Rightarrow e_1$ ; right y $\Rightarrow e_2$)} \to_l e_2\ [y \mapsto v] }

\end{mathpar}

\subsection{NetIR}
\begin{mathpar}
\textsf{fst} (E_1, E_2)  \to E_1 \and
 \textsf{snd} (E_1, E_2)  \to E_2 \and
\textsf{(match \ inl \ E \ with \ inl X} \Rightarrow E_1 \textsf{; inr Y} \Rightarrow E_2) \to E_1\ [X \mapsto E] \and
\textsf{(match \ inr \ E \ with \ inl X} \Rightarrow E_1 \textsf{; inr Y} \Rightarrow E_2) \to E_2\ [Y \mapsto E]
\end{mathpar}
\\
\subsection{Choreography}
\begin{syntax}
     \category[Choreographic Values]{V}
     \alternative{()}
    \alternative{\ell.v}
    \alternative{(V_1, V_2)} 
    \alternative{\textbf{left } V}
    \alternative{\textbf{right } V}
    \alternative{\textbf{fun}\  F \ X \Rightarrow C}
\end{syntax}

\begin{mathparpagebreakable}
\inferrule [assoc ]
{ e \to_l e' }
{ \ell.e \to_g \ell.e' } \and

\inferrule [send 1 ]
{ C \to_g C'}
{ C \leadsto \ell \to_g C' \leadsto \ell} \and
\inferrule [send 2]
{ }
{ \ell.v \leadsto \ell' \to_g \ell'.v} \and
\inferrule [sync ]
{ }
{ \ell_1[d] \leadsto \ell_2 ; C\to_g C} \and
\inferrule[let 1]
{C_1 \to_g C'_1}
{\textsf{let $X := C_1$ in $C_2$} \to_g \textsf{let $X := C'_1$ in $C_2$} } \and
\inferrule[let 2]
{ }
{\textsf{let $ X := V$ in $C$} \to_g C[X \mapsto V] } \and
\inferrule[If 1]
{C \to_g C'}
{\textsf{if } C \textsf{ then } C_1 \textsf{ else } C_2 \to_g \textsf{if } C' \textsf{ then } C_1 \textsf{ else } C_2} \and
\inferrule[If 2]
{ }
{\textsf{if } \ell.true \textsf{ then } C_1 \textsf{ else } C_2 \to_g C_1} \and
\inferrule[If 3]
{ }
{\textsf{if } \ell.false \textsf{ then } C_1 \textsf{ else } C_2 \to_g C_2} \and
\inferrule [App 1]
{ C_1 \to_g C'_1 }
{ C_1 C_2 \to_g C'_1 C_2  } \and
\inferrule [app 2 ]
{ C_2 \to_g C'_2 }
{ \textsf{(fun $F \ X \Rightarrow$ C) } C_2 \to_g \textsf{(fun $F \  X \Rightarrow$ C) } C'_2  } \and
\inferrule [app 3 ]
{  }
{\textsf{(fun$\ F \ X \Rightarrow$ C) }  V \to_g C[X \mapsto V, F \mapsto \textsf{(fun$\ F \ X \Rightarrow$ C) }]  } \and
\inferrule [pair 1]
{ C_1 \to_g C'_1 }
{ (C_1, C_2) \to_g (C'_1, C_2)  } \and
\inferrule [pair 2 ]
{ C_2 \to_g C'_2    }
{ (V, C_2) \to_g (V, C'_2)  } \and

\inferrule [fst ]
{ C \to_g C' }
{ \textsf{fst } C \to_g \textsf{fst } C' } \and
\inferrule [pair elim 1 ]
{ }
{ \textsf{fst } (V_1, V_2) \to_g V_1 } \and
\inferrule [snd ]
{ C \to_g C' }
{ \textsf{snd } C \to_g \textsf{snd } C' } \and
\inferrule [ pair elim 2]
{  }
{ \textsf{snd } (V_1, V_2) \to_g V_2 } \and
\inferrule [left ]
{ C \to_g C' }
{ \textsf{left } C \to_g \textsf{left } C' } \and
\inferrule [right ]
{ C \to_g C' }
{ \textsf{right } C \to_g \textsf{right } C' } \and
\inferrule [match ]
{ C \to_g C' }
{ \textsf{(match C with left X $\Rightarrow C_1$ ; right Y $\Rightarrow C_2$)} \to_g \textsf{(match C' with left X $\Rightarrow C_1$ ; right Y $\Rightarrow C_2$)} } \and
\inferrule [sum elim 1 ]
{ }
{ \textsf{(match (left V) with left X $\Rightarrow C_1$ ; right Y $\Rightarrow C_2$)} \to_g C_1\ [X \mapsto V] } \and
\inferrule [sum elim 2 ]
{ }
{ \textsf{(match (right V) with left X $\Rightarrow C_1$ ; right Y $\Rightarrow C_2$)} \to_g C_2\ [Y \mapsto V] }

\end{mathparpagebreakable}


\section{Theorems}
\textbf{Theorem 1.} (\textit{Local Progress}): For every expression $\cdot \vdash e : t$ either $  \exists \ e'.\ e \to_l e'$  or $e$ is a value
  \\
\textit{Proof.} We will start with induction on e \\\\
\textbf{Case e = ()} \\
() is a value and we are done \\ \\
\textbf{Case e = num} \\
num is a value and we are done \\ \\
\textbf{Case e = $e_1 \odot e_2$} \\
$e_1\ $ and $e_2\ $ are well-typed \\
IH1 $e_1$ is either a value or $  \exists \ e'_1.\ e_1 \to_l e'_1 $ \\
if $e_1 \to_l e'_1$ then $e_1 \odot e_2 \to_l e'_1 \Tilde{\odot} \ e_2$ using \textsc{binop 1} rule\\
if $e_1$ is a value $v_1$, IH2 $e_2$ is either a value or $  \exists \ e'_2.\ e_2 \to_l e'_2 $ \\
if $e_2$ is a value $v_2$, then $v_1 \odot v_2 \to_l v_1 \Tilde{\odot} \ v_2$, which is a value using the \textsc{binop 3} rule  \\
if $e_2 \to_l e'_2$ then $v_1 \odot e_2 \to_l v_1 \Tilde{\odot} \ e'_2$ using \textsc{binop 2} rule 
\\ \\
\textbf{Case e = let $x := e_1$ in $e_2$} \\
$\cdot \vdash e_1 : t$ \\
IH $e_1$ is either a value or $  \exists \ e'_1.\ e_1 \to_l e'_1 $ \\
if $e_1 \to_l e'_1$ then let $x := e_1$ in $e_2 \to_l $ let $x := e'_1$ in $e_2$ using \textsc{loc - let 1} rule\\
if $e_1$ is a value $v$, then let $x := v$ in $e_2 \to_l e_2[x \mapsto v]$ using \textsc{loc - let 2} rule
\\ \\
\textbf{Case e = $(e_1,e_2)$} \\
$\cdot \vdash e_1 : t \quad \cdot \vdash e_2 : t_2$\\
IH1 $e_1$ is either a value or $  \exists \ e'_1.\ e_1 \to_l e'_1 $ \\
if $e_1 \to_l e'_1$ then $(e_1,e_2) \to_l (e'_1, e_2)$ using \textsc{loc - pair 1} rule\\
if $e_1$ is a value $v_1$, IH2 $e_2$ is either a value or $  \exists \ e'_2.\ e_2 \to_l e'_2 $ \\
if $e_2$ is a value $v_2$, then $(v_1,v_2)$ is a value \\
if $e_2 \to_l e'_2$ then $(v_1,e_2) \to_l (v_1,e'_2)$ using \textsc{loc - pair 2} rule \\ \\
\textbf{Case e = fst $e_1$} \\
$\cdot \vdash e_1: t_1 \times t_2$\\
IH $e_1$ is either a value or $  \exists \ e'_1.\ e_1 \to_l e'_1 $ \\
if $e_1 \to_l e'_1$ then fst $e_1 \to_l $ fst $e'_1$ using \textsc{loc - fst} rule\\
if $e_1$ is a value $v$, and $\cdot \vdash e_1: t_1 \times t_2$, this means $v = (v_1, v_2)$ then fst $(v_1, v_2) \to_l v_1$ using \textsc{loc - pair elim 1} rule\\ \\
\textbf{Case e = snd $e_1$} \\
$\cdot \vdash e_1: t_1 \times t_2$\\
IH $e_1$ is either a value or $  \exists \ e'_1.\ e_1 \to_l e'_1 $ \\
if $e_1 \to_l e'_1$ then snd $e_1 \to_l $ snd $e'_1$ using \textsc{loc - snd} rule\\
if $e_1$ is a value $v$, and $\cdot \vdash e_1: t_1 \times t_2$, this means $v = (v_1, v_2)$ then snd $(v_1, v_2) \to_l v_2$ using \textsc{loc - pair elim 2} rule\\ \\
\textbf{Case e = left $e_1$} \\
$\cdot \vdash e_1: t_1$\\
IH $e_1$ is either a value or $  \exists \ e'_1.\ e_1 \to_l e'_1 $ \\
if $e_1 \to_l e'_1$ then left $e_1 \to_l$ left $e'_1$ using \textsc{loc - left} rule\\
if $e_1$ is a value $v$, then left $v$ is a value\\ \\
\textbf{Case e = right $e_1$} \\
$\cdot \vdash e_1: t_2$\\
IH $e_1$ is either a value or $  \exists \ e'_1.\ e_1 \to_l e'_1 $ \\
if $e_1 \to_l e'_1$ then right $e_1 \to_l$ right $e'_1$ using \textsc{loc - right} rule\\
if $e_1$ is a value $v$, then right $v$ is a value\\ \\
\textbf{Case e = match $e_1$ with left x $\Rightarrow e_2$; right y $\Rightarrow e_3$}\\
$\cdot \vdash e_1: t_1 + t_2$\\
IH $e_1$ is either a value or $  \exists \ e'_1.\ e_1 \to_l e'_1 $ \\
if $e_1 \to_l e'_1$ then match $e_1$ with left x $\Rightarrow e_2$; right y $\Rightarrow e_3 \to_l$ match $e'_1$ with left x $\Rightarrow e_2$; right y $\Rightarrow e_3$ using \textsc{loc - match} rule\\
if $e_1$ is a value, and $\cdot \vdash e_1: t_1 + t_2$ this means $e_1$ is either left $v$ or right $v$\\
if $e_1 = $ left $v$ then match (left $v$) with left x $\Rightarrow e_2$; right y $\Rightarrow e_3 \to_l$ $e_2[x \mapsto v]$ using \textsc{loc - sum elim 1} rule\\
if $e_1 = $ right $v$ then match (right $v$) with left x $\Rightarrow e_2$; right y $\Rightarrow e_3 \to_l$ $e_3[y \mapsto v]$ using \textsc{loc - sum elim 2} rule\\
\\\\ 
\\
\textbf{Theorem 2.} (\textit{Local Preservation}): If $\Gamma \vdash e : t$ and  $e \to e'$ then $\Gamma \vdash e' : t$
  \\
\textit{Proof.} We will start with induction on $\Gamma \vdash e : t$ \\\\
\textbf{Case $\inferrule []
{ \; }
{ \Gamma \vdash \textsf{( ) : unit} } \and$} \\
() doesn't take a step and we are done \\ \\
\textbf{Case $\inferrule []
{ x : t \in \Gamma }
{ \Gamma \vdash x : t  } $}      \\ 
$x$ doesn't take a step and we are done \\ \\
\textbf{Case $\inferrule []
{ \Gamma \vdash e_1 : t_1 \quad \Gamma, x : t \vdash e_2 : t_2 }
{ \Gamma ;\Delta \vdash \textsf{let}\ x := e_1 \ \textsf{in}\ e_2 : t_2  }$}      \\ 
if $e_1 \to e'_1$ then let $ x := e_1$ in $ e_2 \to $ let $ x := e'_1$ in $ e_2$ using the \textsc{Loc - Let 1} rule\\
and $\Gamma \vdash$ let $ x := e_1$ in $ e_2 : t_2$ \\ 
IH if $\Gamma \vdash e_1 : t_1$ and  $e_1 \to e'_1$ then $\Gamma \vdash e'_1 : t_1$ \\
Using IH we can say, let $ x := e'_1$ in $ e_2 : t_2$\\
if $e_1 = v$ then let $ x := e_1$ in $ e_2 \to e_2[x \mapsto v]$ using the \textsc{Loc - Let 2} rule\\
and we know $e_2 : t_2$
\\\\
\textbf{Case $\inferrule []
{ \Gamma \vdash e_1 : t_1 \quad \Gamma \vdash e_2 : t_2}
{ \Gamma \vdash (e_1, e_2) : t_1 \times t_2 }$ } \\
IH1 If $\Gamma \vdash e_1 : t_1$ and  $e_1 \to e'_1$ then $\Gamma \vdash e'_1 : t_1$ \\
if $e_1 \to e'_1$ then $(e_1,e_2) \to (e'_1, e_2)$ using the \textsc{Loc - Pair 1} rule\\
and $\Gamma \vdash (e_1, e_2) : t_1 \times t_2$ \\
Using IH1 we can say $\Gamma \vdash (e'_1, e_2) : t_1 \times t_2$ \\
IH2 If $\Gamma \vdash e_2 : t_2$ and  $e_2 \to e'_2$ then $\Gamma \vdash e'_2 : t_2$ \\
if $e_1 = v$ then $(v,e_2) \to (v, e'_2)$ using the \textsc{Loc - Pair 2} rule\\
and $\Gamma \vdash (v, e_2) : t_1 \times t_2$ \\
Using IH2 we can say $\Gamma \vdash (v, e'_2) : t_1 \times t_2$
\\\\
\textbf{Case $\inferrule []
{ \Gamma \vdash e : t_1 \times t_2 }
{ \Gamma \vdash \textsf{fst} \  e : t_1 }$ } \\
if $e \to e'$
IH If $\Gamma \vdash e : t_1 \times t_2$ and  $e \to e'$ then $\Gamma \vdash e' : t_1 \times t_2$ \\
using \textsc{Loc - fst} rule, fst $e \to$ fst $e'$ and we know $\Gamma \vdash$ fst $e : t_1$ \\
Using IH we can say $\Gamma \vdash$ fst $ e' : t_1$ \\
if $e = v$ This means that $e = (v_1, v_2)$ where $v_1 : t_1$
then using \textsc{Loc - pair elim 1}, fst $(v_1, v_2) \to v_1$ and we know $\Gamma \vdash$ fst $(v_1, v_2) : t_1$ \\
and $v_1: t_1$\\\\
\textbf{Case $\inferrule []
{ \Gamma \vdash e : t_1 \times t_2 }
{ \Gamma \vdash \textsf{snd} \  e : t_2 }$ } \\
if $e \to e'$
IH If $\Gamma \vdash e : t_1 \times t_2$ and  $e \to e'$ then $\Gamma \vdash e' : t_1 \times t_2$ \\
using \textsc{Loc - snd} rule, snd $e \to$ snd $e'$ and we know $\Gamma \vdash$ snd $e : t_2$ \\
Using IH we can say $\Gamma \vdash$ snd $ e' : t_2$ \\
if $e = v$ This means that $e = (v_1, v_2)$ where $v_2 : t_2$
then using \textsc{Loc - pair elim 2}, snd $(v_1, v_2) \to v_2$ and we know $\Gamma \vdash$ snd $(v_1, v_2) : t_2$ \\
and $v_2: t_2$\\\\
\textbf{Case $\inferrule []
{ \Gamma \vdash e : t_1 }
{ \Gamma \vdash \textsf{left} \ e : t_1 + t_2 }$ } \\
IH If $\Gamma \vdash e : t_1$ and  $e \to e'$ then $\Gamma \vdash e' : t_1$ \\
using \textsc{Loc - Left} rule we know, left $e \to$ left $e'$ and $\Gamma \vdash$ left $e : t_1+t_2$ \\
Using IH we can say $\Gamma \vdash$ left $ e' : t_1 + t_2$ \\\\
\textbf{Case $\inferrule []
{ \Gamma \vdash e : t_2 }
{ \Gamma \vdash \textsf{right} \ e : t_1 + t_2 }$ } \\
IH If $\Gamma \vdash e : t_2$ and  $e \to e'$ then $\Gamma \vdash e' : t_2$ \\
using \textsc{Loc - Right} rule we know, right $e \to$ right $e'$ and $\Gamma \vdash$ right $e : t_1+t_2$ \\
Using IH we can say $\Gamma \vdash$ right $ e' : t_1 + t_2$ \\\\
\textbf{Case $\inferrule []
{ \Gamma \vdash e : t_1+t_2 \quad \Gamma , \ x : t_1 \vdash e_1 : t_3 \quad \Gamma , \ y : t_2 \vdash e_2 : t_3  }
{ \Gamma \vdash (\textsf{match e with left} \ x \Rightarrow e_1 ; \textsf{right} \ y \Rightarrow e_2) :t_3  } $ } \\
if $e \to e'$
IH If $\Gamma \vdash e : t_1+t_2$ and  $e \to e'$ then $\Gamma \vdash e' : t_1+t_2$ \\
Using \textsc{Loc - match} rule we know, \\ 
match $e$ with left $x \Rightarrow e_2 $; right $ y \Rightarrow e_3 \to$
match $e'$ with left $x \Rightarrow e_2 $; right $ y \Rightarrow e_3$ \\ and $\Gamma \vdash$ match $e$ with left $x \Rightarrow e_2 $; right $ y \Rightarrow e_3 : t_3$ \\
Using IH we can say $\Gamma \vdash$ match $e'$ with left $x \Rightarrow e_2 $; right $ y \Rightarrow e_3 : t_3$ \\
if $e$ is a value then $e =$ left $v$ or right $v$ then using \textsc{Loc - sum elim 1} and \textsc{Loc - sum elim 2} rule respectively, \\
match $e$ with left $x \Rightarrow e_2 $; right $ y \Rightarrow e_3 \to  e_1\ [x \mapsto v] $ or $  e_2\ [y \mapsto v]$ where e = left v or right v\\
and $e_1, e_2 : t_3$
\\\\
\\
\textbf{Theorem 3.} (\textit{Progress}): For every choreography $\cdot \vdash C : \tau$ either $  \exists \ C'.\ C \to_g C'$  or $C$ is a value
  \\
\textit{Proof.} We will start with induction on C \\\\
\textbf{Case C = ()} \\
() is a value and we are done \\ \\
\textbf{Case C = $\ell.e$} \\
$\cdot \vdash e: t$\\
we know from local progress that e is either a value or $ \exists \ e'.\ e \to_l e'$  \\
So, if e is a value $v$ then $C = \ell.v$ which is a value and we are done \\
If $ \exists \ e'.\ e \to_l e'$ then, $ \ell. e \to_g \ell.e'$ using \textsc{assoc} rule
\\ \\
\textbf{Case C = $C_1 \leadsto \ell$} \\
$\cdot \vdash C_1: \ell'.t$\\
IH $C_1$ is either a value or $  \exists \ C'_1.\ C_1 \to_g C'_1 $ \\
if $C_1 \to_g C'_1$ then $C_1 \leadsto \ell \to_g C'_1 \leadsto \ell$ using the \textsc{send 1} rule \\
if $C_1$ is a value and $\cdot \vdash C_1: \ell'.t \ $then, $C_1 = \ell'.v$ and $\ell'.v \leadsto \ell \to_g \ell.v$ using \textsc{send 2} rule 
\\
\\
\textbf{Case C = $\ell_1[d] \leadsto \ell_2; C_1$} \\ 
 we know that, $\ell_1[d] \leadsto \ell_2; C_1 \to_g C_1$ using \textsc{sync} rule \\
\\
\textbf{Case C = if $C_1$ then $C_2$ else $C_3$} \\
$\cdot \vdash C_1: \ell.bool$\\
IH1 $C_1$ is either a value or $  \exists \ C'_1.\ C_1 \to_g C'_1 $ \\
if $C_1 \to C'_1$ then if $C_1$ then $C_2$ else $C_3 \to_g $ if $C'_1$ then $C_2$ else $C_3$ using \textsc{if 1} rule\\
if $C_1$ is a value and $\cdot \vdash C_1: \ell.bool$, then $C_1$ is either $\ell.true$ or $\ell.false$ \\
if $C_1 = \ell.true$ then if $C_1$ then $C_2$ else $C_3 \to_g C_2$ using \textsc{if 2} rule\\
if $C_1 = \ell.false$ then if $C_1$ then $C_2$ else $C_3 \to_g C_3$ using \textsc{if 3} rule\\
\\
\textbf{Case C = let $X := C_1$ in $C_2$} \\
$\cdot \vdash C_1: \tau_1$\\
IH $C_1$ is either a value or $  \exists \ C'_1.\ C_1 \to_g C'_1 $ \\
if $C_1 \to_g C'_1$ then let $X := C_1$ in $C_2 \to_g $ let $X := C'_1$ in $C_2$ using \textsc{let 1} rule\\
if $C_1$ is a value $V$, then let $X := V$ in $C_2 \to_g C_2[X \mapsto V]$ using \textsc{let 2} rule
\\ \\
\textbf{Case C = fun$\ F\ X \Rightarrow C_1$ } \\
fun$\ F\ X \Rightarrow C_1$ is a value and we are done 
\\\\
\textbf{Case C = $C_1 C_2$} \\
$\cdot \vdash C_1: \tau_1 \to \tau_2\ $and$\cdot \vdash C_2: \tau_1$\\
IH1 $C_1$ is either a value or $  \exists \ C'_1.\ C_1 \to_g C'_1 $ \\
if $C_1 \to_g C'_1$ then $C_1 C_2 \to_g C'_1 C_2$ using \textsc{app 1} rule\\
if $C_1$ is a value and $\cdot \vdash C_1: \tau_1 \to \tau_2\ $then$\ C_1 =\ $fun$\ F\ X \Rightarrow C'$,\\
IH2 $C_2$ is either a value or $  \exists \ C'_2.\ C_2 \to_g C'_2 $ \\
if $C_2$ is a value $V$, then (fun$\ F\ X \Rightarrow C') V \to_g C'[X \mapsto V]$ using \textsc{app 3} rule \\
if $C_2 \to_g C'_2$ then (fun$\ F\ X \Rightarrow C') C_2 \to_g$ (fun$\ F\ X \Rightarrow C') C'_2$ using \textsc{app 2} rule 
\\ \\
\textbf{Case C = $(C_1,C_2)$} \\
$\cdot \vdash C_1: \tau_1\ $and$\ \cdot \vdash C_2: \tau_2\ $\\
IH1 $C_1$ is either a value or $  \exists \ C'_1.\ C_1 \to_g C'_1 $ \\
if $C_1 \to_g C'_1$ then $(C_1,C_2) \to_g (C'_1, C_2)$ using \textsc{pair 1} rule\\
if $C_1$ is a value $V_1$, IH2 $C_2$ is either a value or $  \exists \ C'_2.\ C_2 \to_g C'_2 $ \\
if $C_2$ is a value $V_2$, then $(V_1,V_2)$ is a value \\
if $C_2 \to_g C'_2$ then $(V_1,C_2) \to (V_1,C'_2)$ using \textsc{pair 2} rule \\ \\ \\
\textbf{Case C = fst $C_1$} \\
$\cdot \vdash C_1: \tau_1 \times \tau_2\ $\\
IH $C_1$ is either a value or $  \exists \ C'_1.\ C_1 \to_g C'_1 $ \\
if $C_1 \to_g C'_1$ then, fst $C_1 \to_g$ fst $C'_1$ using \textsc{fst} rule\\
if $C_1$ is a value $V$ and $\cdot \vdash C_1: \tau_1 \times \tau_2\ $ this means $V = (V_1, V_2)$ \\ 
then, fst $(V_1, V_2) \to_g V_1$ using \textsc{pair elim 1} rule\\ \\
\textbf{Case C = snd $C_1$} \\
$\cdot \vdash C_1: \tau_1 \times \tau_2\ $\\
IH $C_1$ is either a value or $  \exists \ C'_1.\ C_1 \to_g C'_1 $ \\
if $C_1 \to_g C'_1$ then, snd $C_1 \to_g$ snd $C'_1$ using \textsc{snd} rule\\
if $C_1$ is a value $V$ and $\cdot \vdash C_1: \tau_1 \times \tau_2\ $ this means $V = (V_1, V_2)$ \\ 
then, snd $(V_1, V_2) \to_g V_2$ using \textsc{pair elim 2} rule\\ \\
\textbf{Case C = left $C_1$} \\
$\cdot \vdash C_1: \tau_1$\\
IH $C_1$ is either a value or $  \exists \ C'_1.\ C_1 \to_g C'_1 $ \\
if $C_1 \to_g C'_1$ then left $C_1 \to_g$ left $C'_1$ using \textsc{left} rule\\
if $C_1$ is a value $V$, then left $V$ is a value\\ \\
\textbf{Case C = right $C_1$} \\
$\cdot \vdash C_1: \tau_2$ \\
IH $C_1$ is either a value or $  \exists \ C'_1.\ C_1 \to_g C'_1 $ \\
if $C_1 \to_g C'_1$ then right $C_1 \to_g$ right $C'_1$ using \textsc{right} rule\\
if $C_1$ is a value $V$, then right $V$ is a value\\ \\
\textbf{Case C = match $C_1$ with left X $\Rightarrow C_2$; right Y $\Rightarrow C_3$}\\
$\cdot \vdash C_1: \tau_1 + \tau_2$\\
IH $C_1$ is either a value or $  \exists \ C'_1.\ C_1 \to_g C'_1 $ \\
if $C_1 \to_g C'_1$ then match $C_1$ with left X $\Rightarrow C_2$; right Y $\Rightarrow C_3 \to_g$ match $C'_1$ with left X $\Rightarrow C_2$; right Y $\Rightarrow C_3$ using \textsc{match} rule\\
if $C_1$ is a value and $\cdot \vdash C_1: \tau_1 + \tau_2\ $then $C_1$ is either left $V$ or right $V$\\
if $C_1 = $ left $V$ then match (left $V$) with left X $\Rightarrow C_2$; right Y $\Rightarrow C_3 \to_g$ $C_2[X \mapsto V]$ using \textsc{sum elim 1} rule\\
if $C_1 = $ right $V$ then match (right $V$) with left X $\Rightarrow C_2$; right Y $\Rightarrow C_3 \to_g$ $C_3[Y \mapsto V]$ using \textsc{sum elim 2} rule\\
\\
\\
\textbf{Theorem 4.} (\textit{Preservation}): If $\Gamma;\Delta \vdash C : \tau$ and  $C \to C'$ then $\Gamma;\Delta \vdash C' : \tau$
  \\
\textit{Proof.} We will start with induction on $\Gamma;\Delta \vdash C : \tau$ \\ \\
\textbf{Case $\inferrule []
{ \; }
{ \Gamma ; \Delta \vdash \textsf{( ) : unit} }$} \\
() doesn't take a step and we are done \\ \\
\textbf{Case $\inferrule []
{ X : \tau \in \Delta }
{ \Gamma ;\Delta \vdash X : \tau }$}       \\ 
X doesn't take a step and we are done \\ \\
\textbf{Case $\inferrule []
{ \Gamma_{|\ell} \vdash e : t }
{ \Gamma ; \Delta \vdash \ell.e : \ell.t }$}       \\ 
Using local preservation we can say that, if $\Gamma \vdash e : t$ and  $e \to e'$ then $\Gamma \vdash e' : t$
Now using \textsc{Assoc} rule we know, if $e \to e'$ then $\ell.e \to \ell.e'$ and $\Gamma; \Delta \vdash \ell.e : \ell.t$  \\
Using local preservation, $\Gamma;\Delta \vdash \ell.e' : \ell.t$
\\ \\
\textbf{Case $\inferrule []
{ \Gamma ; \Delta \vdash C : \ell.t}
{ \Gamma;\Delta \vdash C \leadsto \ell_2 : \ell_2.t}$ } \\
if $C \to C'$ \\
IH if $\Gamma;\Delta \vdash C : \tau$ and  $C \to C'$ then $\Gamma; \Delta \vdash C' : \tau$ \\
Using \textsc{Send 1} rule we can say, $ C \leadsto \ell_2 \to  C' \leadsto \ell_2 $\\
using the IH and the \textsc{Send} typing rule, $C' \leadsto \ell_2 : \ell_2.t$\\
if C is a value, then $C = \ell.v$ and using the \textsc{Send 2} rule we can say, \\
$\ell.v \leadsto \ell_2 \to \ell_2.v $ there $\ell.v : \ell.t$ \\
and so, $\ell_2.v : \ell.t$
\\\\
\textbf{Case $\inferrule []
{ \Gamma;\Delta \vdash C : \tau}
{ \Gamma;\Delta \vdash \ell_1[d] \leadsto \ell_2 ; C : \tau}$ } \\
$\ell_1[d] \leadsto \ell_2 ; C : \tau$ and using the \textsc{Sync} rule we know, \\
$\ell_1[d] \leadsto \ell_2 ; C \to C$ and as we know $C : \tau$
\\\\
\textbf{Case $\inferrule []
{ \Gamma;\Delta \vdash C : \ell.bool \quad \Gamma;\Delta\vdash C_2 : \tau_2 \quad \Gamma;\Delta \vdash C_3 : \tau_2 }
{ \Gamma;\Delta \vdash \textsf{if } C \textsf{ then } C_2 \textsf{ else } C_3: \tau_2  }$ } \\
if $C \to C'$ \\
IH if $\Gamma;\Delta \vdash C : \tau$ and  $C \to C'$ then $\Gamma; \Delta \vdash C' : \tau$ \\
Using \textsc{If 1} rule we can say that if $C$ then $C_2$ else $C_3 \to$ if $C'$ then $C_2$ else $C_3$ \\
Using the IH and the \textsc{if} typing rule, if $C'$ then $C_2$ else $C_3 : \tau_2$ \\
If C is a value then $C = \ell.true$ or $ \ell.false$ as $C : \ell.bool$\\
Using the \textsc{if 2} and \textsc{if 3},  if $C$ then $C_2$ else $C_3 \to C_2$ or $C_3$ respectively \\
and we know, $C_2, C_3 : \tau_2$
\\\\
\textbf{Case $\inferrule []
{ \Gamma;\Delta \vdash C_1 : \ell.t \quad \Gamma;\Delta, X : \ell.t \vdash C_2 : \tau_2 }
{ \Gamma;\Delta \vdash \textsf{let } X = C_1 \textsf{ in } C_2 : \tau_2 }$ } \\
if $C_1 \to C'_1$ \\
let $ X := C_1$ in $ C_2 \to $ let $ X := C'_1$ in $ C_2$ using the \textsc{Let 1} rule\\
and $\Gamma; \Delta \vdash$ let $ X := C_1$ in $ C_2 : \tau_2$ \\ 
IH if $\Gamma;\Delta \vdash C_1 : \tau_1$ and  $C_1 \to C'_1$ then $\Gamma; \Delta \vdash C'_1 : \tau_1$ \\
Using IH we can say, let $ X := C'_1$ in $ C_2 : \tau_2$\\
if $C_1 = V$ then let $ X := C_1$ in $ C_2 \to C_2[X \mapsto V]$ using the \textsc{Let 2} rule\\
and we know $C_2 : \tau_2$
\\\\
\textbf{Case $\inferrule []
{ \Gamma;\Delta \vdash C_1 : \tau_1 \to \tau_2 \quad \Gamma;\Delta \vdash C_2 : \tau_1 }
{ \Gamma;\Delta \vdash C_1 C_2 : \tau_2  }$ } \\
if $C_1 \to C'_1$ \\
IH1 if $\Gamma;\Delta \vdash C_1 : \tau_1 \to \tau_2$ and  $C_1 \to C'_1$ then $\Gamma; \Delta \vdash C' : \tau_1 \to \tau_2$ \\
Using the \textsc{App 1} rule, $C_1 C_2 \to C'_1 C_2$ \\
Using the IH1 and \textsc{app} typing rule, $C'_1 C_2 : \tau_2$ \\
If $C_1$ is a value then, $ C_1 =$ fun $F \ X \Rightarrow C$ as $C_1 : \tau_1 \to \tau_2$ \\
Now if $C_2 \to C'_2$ \\
IH2 If $\Gamma;\Delta \vdash C_2 : \tau_1$ and  $C_2 \to C'_2$ then $\Gamma \vdash C'_2 : \tau_1$ \\
Using \textsc{App 2} rule we know, (fun $F \ X \Rightarrow C) C_2 \to ($fun $F \ X \Rightarrow C) C'_2$ \\
And Using IH2 and \textsc{app} typing rule, (fun $F \ X \Rightarrow C) C_2 : \tau_2$ \\
If $C_2$ is a value then $C_2 = V$ \\
Using \textsc{App 3} rule we know,\\
(fun $F \ X \Rightarrow C) V \to C[X \mapsto V, F \mapsto$ (fun $F \ X \Rightarrow C)]$ \\
Using \textsc{Fun} typing rule we can say, $C[X \mapsto V, F \mapsto$ (fun $F \ X \Rightarrow C)] : \tau_2$
\\\\
\textbf{Case $\inferrule []
{ \Gamma ; \Delta \vdash C_1 : \tau_1 \quad \Gamma ; \Delta \vdash C_2 : \tau_2 }
{ \Gamma ; \Delta \vdash (C_1, C_2) : \tau_1 \times \tau_2  }$ } \\
IH1 If $\Gamma;\Delta \vdash C_1 : \tau_1$ and  $C_1 \to C'_1$ then $\Gamma \vdash C'_1 : \tau_1$ \\
if $C_1 \to C'_1$ then $(C_1,C_2) \to (C'_1, C_2)$ using the \textsc{Pair 1} rule\\
and $\Gamma;\Delta \vdash (C_1, C_2) : \tau_1 \times \tau_2$ \\
Using IH1 we can say $\Gamma;\Delta \vdash (C'_1, C_2) : \tau_1 \times \tau_2$ \\
IH2 If $\Gamma;\Delta \vdash C_2 : \tau_2$ and  $C_2 \to C'_2$ then $\Gamma;\Delta \vdash C'_2 : \tau_2$ \\
if $C_1 = V$ then $(V,C_2) \to (V, C'_2)$ using the \textsc{Pair 2} rule\\
and $\Gamma;\Delta \vdash (V, C_2) : \tau_1 \times \tau_2$ \\
Using IH2 we can say $\Gamma;\Delta \vdash (V, C'_2) : \tau_1 \times \tau_2$
\\\\
\textbf{Case $\inferrule []
{ \Gamma ; \Delta \vdash C : \tau_1 \times \tau_2 }
{ \Gamma ; \Delta \vdash \textsf{fst}\ C : \tau_1 }$ } \\
if $C \to C'$
IH If $\Gamma;\Delta \vdash C : \tau_1 \times \tau_2$ and  $C \to C'$ then $\Gamma;\Delta \vdash C' : \tau_1 \times \tau_2$ \\
using \textsc{fst} rule, fst $C \to$ fst $C'$ and we know $\Gamma;\Delta \vdash$ fst $C : \tau_1$ \\
Using IH we can say $\Gamma;\Delta \vdash$ fst $ C' : \tau_1$ \\
if $C = V$ This means that $C = (V_1, V_2)$ where $V_1 : \tau_1$
then using \textsc{pair elim 1}, fst $(V_1, V_2) \to V_1$ and we know $\Gamma;\Delta \vdash$ fst $(V_1, V_2) : \tau_1$ \\
and $V_1: \tau_1$\\\\
\textbf{Case $\inferrule []
{ \Gamma ; \Delta \vdash C : \tau_1 \times \tau_2 }
{ \Gamma ; \Delta \vdash \textsf{snd}\ C : \tau_2 }$ } \\
if $C \to C'$
IH If $\Gamma;\Delta \vdash C : \tau_1 \times \tau_2$ and  $C \to C'$ then $\Gamma;\Delta \vdash C' : \tau_1 \times \tau_2$ \\
using \textsc{snd} rule, snd $C \to$ snd $C'$ and we know $\Gamma;\Delta \vdash$ snd $C : \tau_2$ \\
Using IH we can say $\Gamma;\Delta \vdash$ snd $ C' : \tau_2$ \\
if $C = V$ This means that $C = (V_1, V_2)$ where $V_2 : \tau_2$
then using \textsc{pair elim 2}, snd $(V_1, V_2) \to V_2$ and we know $\Gamma;\Delta \vdash$ snd $(V_1, V_2) : \tau_2$ \\
and $V_2: \tau_2$\\\\
\textbf{Case $\inferrule []
{ \Gamma ; \Delta \vdash C : \tau_1 }
{ \Gamma ; \Delta \vdash \textsf{left}\ C : \tau_1 + \tau_2 }$ } \\
IH If $\Gamma;\Delta \vdash C : \tau_1$ and  $C \to C'$ then $\Gamma;\Delta \vdash C' : \tau_1$ \\
using \textsc{Left} rule we know, left $C \to$ left $C'$ and $\Gamma;\Delta \vdash$ left $e : \tau_1+\tau_2$ \\
Using IH we can say $\Gamma;\Delta \vdash$ left $ C' : \tau_1 + \tau_2$ \\\\
\textbf{Case $\inferrule []
{ \Gamma ; \Delta \vdash C : \tau_2 }
{ \Gamma ; \Delta \vdash \textsf{right}\ C : \tau_1 + \tau_2 }$ } \\
IH If $\Gamma;\Delta \vdash C : \tau_2$ and  $C \to C'$ then $\Gamma;\Delta \vdash C' : \tau_2$ \\
using \textsc{Right} rule we know, right $C \to$ right $C'$ and $\Gamma;\Delta \vdash$ right $C : \tau_1+\tau_2$ \\
Using IH we can say $\Gamma;\Delta \vdash$ right $ C' : \tau_1 + \tau_2$ \\\\
\textbf{Case $\inferrule []
{ \Gamma ; \Delta \vdash C : \tau_1+\tau_2 \quad \Gamma ; \Delta , X : \tau_1 \vdash C_1 : \tau_3 \quad \Gamma ; \Delta , Y : \tau_2 \vdash C_2 : \tau_3  }
{ \Gamma ; \Delta \vdash (\textsf{match}\ C \ \textsf{with left}\ X \Rightarrow C_1 ; \textsf{right}\ Y \Rightarrow C_2) : \tau_3 } $ } \\
if $C \to C'$
IH If $\Gamma;\Delta \vdash C : \tau_1+\tau_2$ and  $C \to C'$ then $\Gamma;\Delta \vdash C' : \tau_1+\tau_2$ \\
Using \textsc{match} rule we know, \\ 
match $C$ with left $X \Rightarrow C_2 $; right $ Y \Rightarrow C_3 \to$
match $C'$ with left $X \Rightarrow C_2 $; right $ Y \Rightarrow C_3$ \\ and $\Gamma;\Delta \vdash$ match $C$ with left $X \Rightarrow C_2 $; right $ Y \Rightarrow C_3 : \tau_3$ \\
Using IH we can say $\Gamma;\Delta \vdash$ match $C'$ with left $X \Rightarrow C_2 $; right $ Y \Rightarrow C_3 : \tau_3$ \\
if $C$ is a value then $C =$ left $V$ or right $V$ then using \textsc{sum elim 1} and \textsc{sum elim 2} rule respectively, \\
match $C$ with left $V \ X \Rightarrow C_2 $; right $ Y \Rightarrow C_3 \to  C_2\ [X \mapsto V] $ or $  C_3\ [Y \mapsto V]$ where C = left V or right V\\
and $C_2, C_3 : \tau_3$ \\\\



\section{Glossary}
$$
\ell \textsf{ involved in } \tau = 
    \begin{array}{l}
    \ell \in locs (\tau)
    \end{array}
$$
$$
 \ell \in locs (\tau) = \textsf{ getLoc is a function that recursively traverses over } \tau \textsf{ to construct } locs (\tau)
$$

$$
 locs (\tau) = \left\{
    \begin{array}{ll}
    \phi & \textsf{if } \tau = \textbf{unit}\\
    \{\ell\} & \textsf{if } \tau = \ell.e\\
        \textsf{getLoc } \tau_1 \cup \textsf{getLoc } \tau_2 & \textsf{if } \tau = \tau_1 \to \tau_2 \textsf{ or } \tau_1 + \tau_2 \textsf{ or } \tau_1 \times \tau_2 \\
    \end{array}
\right.
$$

\section{Endpoint Projection}

$$
 \llbracket     (C_1, \ C_2) \rrbracket _\ell = \left\{
    \begin{array}{l l}
    (\llbracket     C_1 \rrbracket _\ell ,\  \llbracket     C_2 \rrbracket _\ell)  & 
    \textsf{if } (C_1, C_2) : \tau_1 \times \tau_2 \textsf{ and } \ell \textsf{ is involved in } C_1,  C_2, \tau_1, \ \tau_2 \\
    \begin{array}{l}
    \textsf{let x = } \llbracket C_1 \rrbracket _\ell \textsf{ in} \\
    \textsf{let \_ = } \llbracket C_2 \rrbracket _\ell \textsf{ in x}
    \end{array} & \textsf{if } (C_1, C_2) : \tau_1 \times \tau_2 \textsf{ and } \ell \textsf{ is involved in } C_1, C_2 \textsf{ and } \tau_1 \textsf{ but not in } \tau_2  \\
    \llbracket     C_1 \rrbracket _\ell; \llbracket     C_2 \rrbracket _\ell & \textsf{if } (C_1, C_2) : \tau_1 \times \tau_2 \textsf{ and } \ell \textsf{ is involved in }  C_1 \textsf{ or }  C_2 \\ 
    \textbf{()} & \textsf{otherwise}
    \end{array}
\right.
$$

$$
 \llbracket  \textbf{fst } C \rrbracket _\ell = \left\{
    \begin{array}{ll}
    \textbf{fst }\llbracket     C \rrbracket _\ell & \textsf{if } C : \tau_1 \times \tau_2 \textsf{ and } \ell \textsf{ is involved in } C,  \tau_1 \textsf{ and } \tau_2\\
    \llbracket C \rrbracket _\ell & \textsf{if } C : \tau_1 \times \tau_2 \textsf{ and } \ell \textsf{ is involved in } C  \\
    \textbf{()} & \textsf{otherwise}
    \end{array}
\right.
$$

$$
 \llbracket  \textbf{snd} C \rrbracket _\ell = \left\{
    \begin{array}{ll}
    \textbf{snd }\llbracket     C \rrbracket _\ell & \textsf{if } C : \tau_1 \times \tau_2 \textsf{ and } \ell \textsf{ is involved in } C,  \tau_1 \textsf{ and } \tau_2\\
    \llbracket C \rrbracket _\ell & \textsf{if } C : \tau_1 \times \tau_2 \textsf{ and } \ell \textsf{ is involved in } C  \\
    \textbf{()} & \textsf{otherwise}
    \end{array}
\right.
$$

$$
 \llbracket  \textbf{inl } C \rrbracket _\ell = \left\{
    \begin{array}{ll}
    \textbf{inl }\llbracket     C \rrbracket _\ell & \textsf{if } C : \tau_1 \textsf{ and } \ell \textsf{ is involved in } C \textsf{ and } \tau_1 \\
    \llbracket     C \rrbracket _\ell & \textsf{if } C : \tau_1 \textsf{ and } \ell \textsf{ is involved in } C \textsf{ but not in } \tau_1 \\
    \textbf{()} & \textsf{if } C : \tau_1 \textsf{ and } \ell \textsf{ is not involved in } C \textsf{ and } \tau_1
    \end{array}
\right.
$$

$$
 \llbracket  \textbf{inr } C \rrbracket _\ell = \left\{
    \begin{array}{ll}
    \textbf{inr }\llbracket     C \rrbracket _\ell & \textsf{if } C : \tau_2 \textsf{ and } \ell \textsf{ is involved in } C \textsf{ and } \tau_2 \\
    \llbracket     C \rrbracket _\ell & \textsf{if } C : \tau_2 \textsf{ and } \ell \textsf{ is involved in } C \textsf{ but not in } \tau_2 \\
    \textbf{()} & \textsf{if } C : \tau_2 \textsf{ and } \ell \textsf{ is not involved in } C \textsf{ and } \tau_2
    \end{array}
\right.
$$

$$\begin{array}{l}
 \llbracket  \textbf{match } C \textbf{ in inl X } \Rightarrow C_1 \textbf{; inr Y } \Rightarrow C_2 \rrbracket _\ell\\ \\ \;\;= \left\{
    \begin{array}{l l}
    \textbf{match }\llbracket C \rrbracket _\ell \textbf{ in inl X } \Rightarrow \llbracket C_1 \rrbracket _\ell \textbf{; inr Y } \Rightarrow \llbracket C_2 \rrbracket _\ell & 
    \begin{array}{l}
    \textsf{if } C : \tau_1 + \tau_2 \textsf{ and } \ell \textsf{ is involved in} \\ \textsf{both } \tau_1 \textsf{ and }\tau_2 
    \end{array}    \\
    \llbracket C \rrbracket _\ell \textbf{; } \llbracket C_1 \rrbracket _\ell \sqcup \llbracket C_2 \rrbracket _\ell & 
    \begin{array}{l}
    \textsf{if } C : \tau_1 + \tau_2 \textsf{ and } \ell \textsf{ is involved in } \\ \tau_1 \textsf{ or } \tau_2 \textsf{ or } C
    \end{array}    \\
    \llbracket C_1 \rrbracket _\ell \sqcup \llbracket C_2 \rrbracket _\ell & 
    \begin{array}{l}
    \textsf{if } C : \tau_1 + \tau_2 \textsf{ and } \ell \textsf{ is not involved in} \\ C, \ \tau_1 \textsf{ and } \tau_2 
    \end{array}
    \end{array}
\right.
\end{array}
$$

\section{Type Projection} 
$$
 \llbracket  \textbf{unit} \rrbracket _\ell =
    \begin{array}{ll}
    \textbf{unit}
    \end{array}
$$

$$
 \llbracket  \ell_1 . t \rrbracket _{\ell_2} = \left\{
    \begin{array}{ll}
    t & \textsf{if } \ell_1 = \ell_2 \\
    \textbf{unit} & \textsf{otherwise}
    \end{array}
\right.
$$

$$
 \llbracket  \tau_1 \to \tau_2 \rrbracket _\ell = \left\{
    \begin{array}{ll}
    \llbracket  \tau_1 \rrbracket _\ell \to \llbracket  \tau_2 \rrbracket _\ell & \textsf{if } \ell \textsf{ is involved in } \tau_1 \textsf{ or } \tau_2 \textsf{ or both}\\
    \textbf{unit} & \textsf{otherwise}
    \end{array}
\right.
$$

$$
 \llbracket  \tau_1 + \tau_2 \rrbracket _\ell = \left\{
    \begin{array}{ll}
    \llbracket  \tau_1 \rrbracket_\ell + \llbracket  \tau_2 \rrbracket _\ell & \textsf{if } \ell \textsf{ is involved in } \tau_1 \textsf{ or } \tau_2 \textsf{ or both}\\
    \textbf{unit} & \textsf{otherwise}
    \end{array}
\right.
$$

$$
 \llbracket  \tau_1 \times \tau_2 \rrbracket _\ell = \left\{
    \begin{array}{ll}
    \llbracket  \tau_1 \rrbracket_\ell \times \llbracket  \tau_2 \rrbracket _\ell & \textsf{if } \ell \textsf{ is involved in } \tau_1 \textsf{ and } \tau_2 \\
    \llbracket  \tau_1 \rrbracket_\ell & \textsf{if } \ell \textsf{ is involved in } \tau_1 \textsf{ but not } \tau_2 \\
    \llbracket  \tau_2 \rrbracket_\ell & \textsf{if } \ell \textsf{ is involved in } \tau_2 \textsf{ but not } \tau_1 \\
    \textbf{unit} & \textsf{otherwise}
    \end{array}
\right.
$$

\section{Lemmas}
$$
    \begin{array}{ll}
    \textbf{Lemma 1 : If } \ell \textbf{ is not involved in } \tau \textbf{ then } \llbracket  \tau \rrbracket_\ell = \textbf{unit}
    \\
    \textbf{Proof : } \textsf{By Induction on } \tau \\
    \\
     \textbf{Case } \tau = \textbf{unit : } \\
   \quad \textsf{From type projection we know, }   \llbracket 
   \textbf{unit} \rrbracket _\ell = \textsf{unit} \\
   \\
   \textbf{Case } \tau = \ell.t\textbf{ : } \\
    \quad \textsf{Here we know that, } \ell_1 \neq \ell \, \  \textsf{as } \ell \textsf{ is not involved in } \tau \\
    \quad \textsf{so using type projection for } \llbracket 
   \ell_1.t\rrbracket _\ell \\ \quad \textsf{we can say, }
   \llbracket 
   \ell_1.t\rrbracket _\ell = \textbf{unit } \\
   \\
   \textbf{Case } \tau = \llbracket  \tau_1 \to \tau_2 \rrbracket _\ell \textbf{: } \\
   \quad \textsf{By IH, } \ell \textsf{ is not involved in } \tau_1 \textsf{ and } \ell \textsf{ is not involved in } \tau_2\\
    \quad \textsf{so using type projection for } \llbracket 
   \tau_1 \to \tau_2\rrbracket _\ell \\ \quad \textsf{we can say, }
   \llbracket 
   \tau_1 \to \tau_2\rrbracket _\ell = \textbf{unit } \\
   \\
   \textbf{Case } \tau = \llbracket  \tau_1 + \tau_2 \rrbracket _\ell \textbf{: } \\
   \quad \textsf{By IH, } \ell \textsf{ is not involved in } \tau_1 \textsf{ and } \ell \textsf{ is not involved in } \tau_2\\
    \quad \textsf{so using type projection for } \llbracket 
   \tau_1 + \tau_2\rrbracket _\ell \\ \quad \textsf{we can say, }
   \llbracket 
   \tau_1 + \tau_2\rrbracket _\ell = \textbf{unit } \\
   \\
   \textbf{Case } \tau = \llbracket  \tau_1 \times \tau_2 \rrbracket _\ell \textbf{: } \\
   \quad \textsf{By IH, } \ell \textsf{ is not involved in } \tau_1 \textsf{ and } \ell \textsf{ is not involved in } \tau_2\\
    \quad \textsf{so using type projection for } \llbracket 
   \tau_1 \times \tau_2\rrbracket _\ell \\ \quad \textsf{we can say, }
   \llbracket 
   \tau_1 * \tau_2\rrbracket _\ell = \textbf{unit } \\
   
    
    \\
    \\
    \textbf{Lemma 2 : If } \ell \notin locs(C) \textbf{ then }\llbracket  C \rrbracket_\ell = \textbf{()}
    \\
    \\
    \textbf{Lemma 3 : If } \vdash C : \tau, \textbf{ then } \vdash \llbracket  C \rrbracket_\ell : \llbracket  \tau \rrbracket_\ell
    \end{array}
$$



\end{document}
